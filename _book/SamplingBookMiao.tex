\documentclass[fontset=fandol,zihao=false,scheme=chinese,heading=true,UTF8]{ctexbook}
\usepackage{lmodern}
\usepackage{amssymb,amsmath}
\PassOptionsToPackage{dvipsnames}{xcolor}
\RequirePackage{xcolor} % [dvipsnames]
%\usepackage{xcolor}

\usepackage{ifxetex,ifluatex}
\usepackage{fixltx2e} % provides \textsubscript
\ifnum 0\ifxetex 1\fi\ifluatex 1\fi=0 % if pdftex
  \usepackage[T1]{fontenc}
  \usepackage[utf8]{inputenc}
\else % if luatex or xelatex
  \ifxetex
    \usepackage{xltxtra,xunicode}
  \else
    \usepackage{fontspec}
  \fi
  \defaultfontfeatures{Ligatures=TeX,Scale=MatchLowercase}
\fi
% use upquote if available, for straight quotes in verbatim environments
\IfFileExists{upquote.sty}{\usepackage{upquote}}{}
% use microtype if available
\IfFileExists{microtype.sty}{%
\usepackage{microtype}
\UseMicrotypeSet[protrusion]{basicmath} % disable protrusion for tt fonts
}{}
\usepackage[b5paper,tmargin=2.5cm,bmargin=2.5cm,lmargin=2.5cm,rmargin=2.5cm]{geometry}
\usepackage[unicode=true]{hyperref}
\PassOptionsToPackage{usenames,dvipsnames}{color} % color is loaded by hyperref
\hypersetup{
            pdftitle={用R语言做社会调查抽样设计},
            pdfauthor={蔡苗},
            colorlinks=true,
            linkcolor=Maroon,
            citecolor=Blue,
            urlcolor=Blue,
            breaklinks=true}
\urlstyle{same}  % don't use monospace font for urls
\usepackage{natbib}
\bibliographystyle{apalike}
\usepackage{longtable,booktabs}
% Fix footnotes in tables (requires footnote package)
\IfFileExists{footnote.sty}{\usepackage{footnote}\makesavenoteenv{long table}}{}
\usepackage{graphicx,grffile}
\makeatletter
\def\maxwidth{\ifdim\Gin@nat@width>\linewidth\linewidth\else\Gin@nat@width\fi}
\def\maxheight{\ifdim\Gin@nat@height>\textheight\textheight\else\Gin@nat@height\fi}
\makeatother
% Scale images if necessary, so that they will not overflow the page
% margins by default, and it is still possible to overwrite the defaults
% using explicit options in \includegraphics[width, height, ...]{}
\setkeys{Gin}{width=\maxwidth,height=\maxheight,keepaspectratio}
\IfFileExists{parskip.sty}{%
\usepackage{parskip}
}{% else
\setlength{\parindent}{0pt}
\setlength{\parskip}{6pt plus 2pt minus 1pt}
}
\setlength{\emergencystretch}{3em}  % prevent overfull lines
\providecommand{\tightlist}{%
  \setlength{\itemsep}{0pt}\setlength{\parskip}{0pt}}
\setcounter{secnumdepth}{5}
% Redefines (sub)paragraphs to behave more like sections
\ifx\paragraph\undefined\else
\let\oldparagraph\paragraph
\renewcommand{\paragraph}[1]{\oldparagraph{#1}\mbox{}}
\fi
\ifx\subparagraph\undefined\else
\let\oldsubparagraph\subparagraph
\renewcommand{\subparagraph}[1]{\oldsubparagraph{#1}\mbox{}}
\fi

% set default figure placement to htbp
\makeatletter
\def\fps@figure{htbp}
\makeatother

\usepackage{booktabs}
\usepackage{longtable}

\usepackage{framed,color}
\definecolor{shadecolor}{RGB}{248,248,248}
\usepackage{style/lshort-zn-cnMiao}

\renewcommand{\textfraction}{0.05}
\renewcommand{\topfraction}{0.8}
\renewcommand{\bottomfraction}{0.8}
\renewcommand{\floatpagefraction}{0.75}

\newenvironment{dedication}
{
   \cleardoublepage
   \thispagestyle{empty}
   \vspace*{\stretch{1}}
   \hfill\begin{minipage}[t]{0.66\textwidth}
   \raggedright
}
{
   \end{minipage}
   \vspace*{\stretch{3}}
   \clearpage
}

\let\oldhref\href
\renewcommand{\href}[2]{#2\footnote{\url{#1}}}

\makeatletter
\newenvironment{kframe}{%
\medskip{}
\setlength{\fboxsep}{.8em}
 \def\at@end@of@kframe{}%
 \ifinner\ifhmode%
  \def\at@end@of@kframe{\end{minipage}}%
  \begin{minipage}{\columnwidth}%
 \fi\fi%
 \def\FrameCommand##1{\hskip\@totalleftmargin \hskip-\fboxsep
 \colorbox{shadecolor}{##1}\hskip-\fboxsep
     % There is no \\@totalrightmargin, so:
     \hskip-\linewidth \hskip-\@totalleftmargin \hskip\columnwidth}%
 \MakeFramed {\advance\hsize-\width
   \@totalleftmargin\z@ \linewidth\hsize
   \@setminipage}}%
 {\par\unskip\endMakeFramed%
 \at@end@of@kframe}
\makeatother

%\renewenvironment{Shaded}{\begin{kframe}}{\end{kframe}}

\usepackage{makeidx}
\makeindex

\urlstyle{tt}

\usepackage{amsthm}
\makeatletter
\def\thm@space@setup{%
  \thm@preskip=8pt plus 2pt minus 4pt
  \thm@postskip=\thm@preskip
}
\makeatother

\frontmatter


\title{用R语言做社会调查抽样设计}
\author{蔡苗}
\date{2019-04-04}

\begin{document}
\maketitle

\begin{dedication}
感谢我的家人的支持。
\end{dedication}

\section*{Acknowledgement}

I want to thank my mentor.

{
\setcounter{tocdepth}{2}
\tableofcontents
}
\listoftables
\listoffigures



\hypertarget{section}{%
\chapter{序}\label{section}}

在大多数依赖于社会调查进行研究的学科中,最常见遇到的问题就是应该怎么抽样?在那些地方抽样?每个地方抽多少样本?

\cleardoublepage 
\mainmatter

\hypertarget{section-1}{%
\chapter{引言}\label{section-1}}

\begin{itemize}
\item
  \textbf{总体} - 参数(parameter):研究者想要知道但是无法知道的总体特征,比如全国男性平均身高,
\item
  \textbf{样本} - 统计量(statistic):研究者可以通过抽样进行实际计算的样本特征,比如全国普查男性平均身高。
\item
  \textbf{无限总体}: 总体的量非常大,我们可以假设我们抽出来的每个观察值都是独立同分布的。
\item
  \textbf{有限总体}: 总体的量不大,我们抽出来的样本并不独立,但是同分布。
\end{itemize}

假设我们要研究某大学社会学系总共30个班级本科学生的四级考试平均成绩,此时总体可能无法看做是无限总体。
这是因为总共30个班级人数较少,并且不同年级同学四级考试成绩可能会存在系统差异,因此此总体看做是有限总体比较合理。
假设我们要研究某小池塘中的平均汞浓度,我们可以从池塘的不同边缘抽出10小杯样本水,然后检测每个样本中汞浓度。
在此例子中,我们可以假设总体为无限总体,因为小池塘中的汞浓度相对均匀,并且抽取10小杯水并不会对总体池塘的水有太大影响。

\begin{itemize}
\tightlist
\item
  \textbf{简单随机抽样}\index{简单随机抽样}
\item
  \textbf{分层抽样}\index{分层抽样}
\item
  \textbf{聚类抽样}\index{聚类抽样}
\item
  \textbf{系统抽样}\index{系统抽样}
\end{itemize}



\begin{figure}
\centering
\includegraphics{SamplingBookMiao_files/figure-latex/unnamed-chunk-1-1.pdf}
\caption{\label{fig:unnamed-chunk-1}这是一个\textbf{中文标题}。}
\end{figure}

Note that the \texttt{echo\ =\ FALSE} parameter was added to the code chunk to prevent printing of the R code that generated the plot.

\hypertarget{section-2}{%
\chapter{简单随机抽样}\label{section-2}}

\textbf{简单随机抽样}(Simple Random Sample, SRS):总体中的不同单位拥有相同的被抽中几率。

假设我们能够从均值为\(\mu\),方差为\(\sigma^2\)的总体中不放回抽样得到观测值\(y_1, y_2, \dots, y_n\),那么我们可以得到以下关于均数\(\mu\)的点估计量:

\[\hat{\mu} = \bar{y} = \frac{1}{n}\sum_{i=1}^ny_i\]

并且我们可以得到以下性质:

\[
E(\hat{\mu}) = \mu\\
V(\hat{\mu}) = \frac{\sigma^2}{n}\frac{N-n}{N-1}
\]

第一个性质说明此\(\hat{\mu}\)为无偏无计量。

我们同时还可以得到关于方差\(\sigma^2\)的估计量:

\[\hat{\sigma}^2 = s^2 = \frac{1}{n-1}\sum_{i=1}^n(y-\bar{y})^2\]

并且有以下性质:

\[
E(\sigma^2) = \frac{N}{N-1}\sigma^2\\
\hat{V}(\hat{\mu}) = \frac{s^2}{n}(1-\frac{n}{N})
\]

\hypertarget{section-3}{%
\chapter{分层抽样}\label{section-3}}

\hypertarget{section-4}{%
\chapter{比例数据和回归}\label{section-4}}

\hypertarget{section-5}{%
\section{比例数据}\label{section-5}}

\hypertarget{section-6}{%
\section{回归}\label{section-6}}

\hypertarget{section-7}{%
\chapter{聚类抽样}\label{section-7}}

\hypertarget{section-8}{%
\chapter{非均等概率抽样}\label{section-8}}

\hypertarget{section-9}{%
\chapter{大型复杂社会调查}\label{section-9}}

\hypertarget{section-10}{%
\chapter{调查对象拒绝回答}\label{section-10}}

\bibliography{bib/bib.bib}


\backmatter
\printindex

\end{document}
